\documentclass[twocolumn]{aastex63}

% typography
\usepackage[T1]{fontenc}

\usepackage{amsmath}

\setlength{\parindent}{1.\baselineskip}
\newcommand{\acronym}[1]{{\small{#1}}}
\newcommand{\package}[1]{\textsl{#1}}
\newcommand{\gaia}{\textsl{Gaia}}

% \newcommand{\deg}{\ensuremath{\textrm{deg}}}
\newcommand{\gyr}{\ensuremath{\textrm{Gyr}}}
\newcommand{\kpc}{\ensuremath{\textrm{kpc}}}
\newcommand{\mas}{\ensuremath{\textrm{mas}}}
\newcommand{\ul}{\ensuremath{\textrm{kpc}^2\,\textrm{Myr}^{-1}}}
\newcommand{\ue}{\ensuremath{\textrm{kpc}^2\,\textrm{Myr}^{-2}}}
\newcommand{\kms}{\ensuremath{\textrm{km}\,\textrm{s}^{-1}}}
\newcommand{\masyr}{\ensuremath{\textrm{mas}\,\textrm{yr}^{-1}}}
\newcommand{\feh}{\ensuremath{\textrm{[Fe/H]}}}
\newcommand{\afe}{\ensuremath{\textrm{[$\alpha$/Fe]}}}

\newcommand{\changes}[1]{{\textbf{#1}}}

% aastex parameters
% \received{not yet; THIS IS A DRAFT}
%\revised{not yet}
%\accepted{not yet}
% % Adds "Submitted to " the arguement.
% \submitjournal{AJ}
\shorttitle{}
\shortauthors{cats}

%@arxiver{}

\begin{document}\sloppy\sloppypar\raggedbottom\frenchspacing % trust me

\title{Community Atlas of Tidal Streams}

% \correspondingauthor{}
% \email{}

\author{CATS Team}
\affil{Around the globe}

% \author[0000-0002-7846-9787]{Ana~Bonaca}
% \affil{The Observatories of the Carnegie Institution for Science, 813 Santa Barbara Street, Pasadena, CA 91101, USA}


\begin{abstract}\noindent % trust me
Recent years have seen an enormous growth in the discovery of tidal streams in the Milky Way, reaching nearly a hundred known streams today.
This unprecedented data set provides an avenue to study some of the first stars in the Universe and presents some of the most stringent tests of the dark-matter paradigm.
To deliver on these promises, a homogeneous catalog of all known streams is needed.
Here we present a well-motivated series of modeling procedures to homogeneously characterize the known streams in publicly available astrometry, photometry, and spectroscopy.
We demonstrate these procedures on five stellar streams spanning a wide range in length, distance and surface brightness: GD-1, Palomar~5, Jhelum, Fj\" orm, and PS1-A.
The outcome of our modeling are membership probabilities of the constituent stream stars.
All of our code and the catalogs for the five analyzed streams are publicly available, and constitute the first contribution towards the Community Atlas of Tidal Streams (CATS).
CATS will be a lasting asset to the community, making once prohibitively time-expensive projects possible, and significantly easing the entry of newcomers into the field.
\end{abstract}

\section{Introduction}

\section{Datasets}

\section{Workflow / Methods}

\subsection{From discovery to pawprints}

\subsection{Proper motions}

\subsection{Color-magnitude diagram}

\subsection{Density modeling}

\subsection{Distance tracers}

\subsection{Spectroscopy}


\section{Results}
- membership catalogs for the 5 selected streams, in order of difficulty
- based on these catalogs, we also discuss any density variations, abundances, constraints on the stellar mass

\subsection{GD-1}


\subsection{Palomar~5}


\subsection{Jhelum}


\subsection{Fj\" orm}


\subsection{PS1-A}
- a null result also valuable



\section{Discussion}



\section{Conclusions}



\vspace{0.5cm}
It is a pleasure to thank


\software{
\package{Astropy} \citep{astropy, astropy:2018},
\package{gala} \citep{gala},
\package{IPython} \citep{ipython},
\package{matplotlib} \citep{mpl},
\package{numpy} \citep{numpy},
\package{scipy} \citep{scipy}
}

\bibliographystyle{aasjournal}
\bibliography{cats}


\end{document}
